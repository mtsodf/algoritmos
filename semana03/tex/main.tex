\documentclass[aspectratio=169,usenames,dvipsnames]{beamer}

\usetheme{default}  % You can choose any other theme you prefer

\title{03 - Algoritmos}
\author{Mateus Oliveira de Figueiredo}
\date{12/09/2023}

\usepackage{tikz}
\usepackage{multicol}
\usepackage{algorithm}
\usepackage{algpseudocode}
\usepackage{xcolor}
\usepackage[utf8]{inputenc}

\usepackage{pgfplots}
\DeclareUnicodeCharacter{2212}{−}
\usepgfplotslibrary{groupplots,dateplot}
\usetikzlibrary{patterns,shapes.arrows}
\pgfplotsset{compat=newest}

\begin{document}

\begin{frame}
\titlepage
\end{frame}

\begin{frame}
\frametitle{Problema}

\onslide<1->{
    Dado um conjunto de pontos no  $\mathbb{R}^2$, encontrar o menor polígono convexo que contém todos os pontos.
}
\begin{figure}
\begin{overprint}
  \onslide<2> \include{figures/points}
  \onslide<3> \include{figures/points_hull}
\end{overprint}
\end{figure}

\end{frame}

\begin{frame}{Algoritmos}
      Três algoritmos implementados:
      \begin{itemize}
        \item Algoritmo Triângulos: $O(n^4)$
        \item Algoritmo Segmentos: $O(n^3)$
        \item Algoritmo de Jarvis: $O(hn)$
      \end{itemize}
\end{frame}

\begin{frame}{Algoritmo Triângulos}

  % Add centralized figure
  \begin{center}
    \begin{figure}
      \begin{overprint}
        \onslide<1>\include{figures/algs/naive_0}
        \onslide<2>\include{figures/algs/naive_1}
        \onslide<3>\include{figures/algs/naive_2}
        \onslide<4>\include{figures/algs/naive_3}
        \onslide<5>% This file was created with tikzplotlib v0.10.1.
\begin{tikzpicture}[scale=0.9]

\definecolor{darkslategray38}{RGB}{38,38,38}
\definecolor{lightgray204}{RGB}{204,204,204}
\definecolor{steelblue76114176}{RGB}{76,114,176}

\begin{axis}[
axis line style={lightgray204},
hide x axis,
hide y axis,
tick align=outside,
x grid style={lightgray204},
xmajorticks=false,
xmin=0.0669367338947019, xmax=0.894111789939664,
xtick style={color=darkslategray38},
y grid style={lightgray204},
ymajorticks=false,
ymin=-0.0362471743023485, ymax=1.03274876525929,
ytick style={color=darkslategray38}
]
\addplot [draw=steelblue76114176, fill=steelblue76114176, mark=*, only marks]
table{%
x  y
0.346517060199814 0.201667385571862
0.110198245014805 0.710446973195645
0.674196381095166 0.49152149407659
0.30956117598126 0.80506608165215
0.271019344924566 0.98415804073376
0.636154602711755 0.693008958661624
0.384544745815431 0.621686927298183
0.613069361293463 0.536676532843469
0.515872695614497 0.0123435502231805
0.856512923755802 0.429885536698643
0.794765500051758 0.575905243509799
0.104535600078564 0.331190264339077
0.698107527894101 0.591287819523784
0.652331867510728 0.829843579751645
0.776100950068952 0.0753407526360956
0.359461819254946 0.832493157115646
0.234215740614521 0.440279847038206
0.474800803395323 0.180277096021146
0.478504950884847 0.867281705269675
0.434092000130053 0.504691515965982
0.640277876553855 0.402360080889356
};
\addplot [draw=red, fill=red, mark=*, only marks]
table{%
x  y
0.613069361293463 0.536676532843469
0.434092000130053 0.504691515965982
};
\addplot [draw=red, fill=red, mark=*, only marks]
table{%
x  y
0.384544745815431 0.621686927298183
0.234215740614521 0.440279847038206
};
\addplot [semithick, black]
table {%
0.104535600078564 0.331190264339077
0.474800803395323 0.180277096021146
0.478504950884847 0.867281705269675
0.104535600078564 0.331190264339077
};
\end{axis}

\end{tikzpicture}

        \onslide<6->\include{figures/algs/naive_5}
      \end{overprint}
    \end{figure}
  \end{center}

  \begin{overprint}
    \onslide<6>{
      \begin{equation}
        \text{Custo} = \underbrace{O(n^3)}_{\text{Triângulos}} \times \underbrace{O(n)}_{\text{Pontos dentro}} +\underbrace{O(nlog(n))}_{\text{Ordenar pontos}} = O(n^4)
      \end{equation}
    }
  \end{overprint}
\end{frame}

\begin{frame}{Algoritmo Segmentos}
  % Add centralized figure
  \begin{center}
    \begin{figure}
      \begin{overprint}
        \onslide<1>\include{figures/algs/segments_0}
        \onslide<2->\include{figures/algs/segments_1}
      \end{overprint}
    \end{figure}
  \end{center}

  \begin{overprint}
    \onslide<3>{
      \begin{equation}
        \text{Custo} = \underbrace{O(n^2)}_{\text{Segmentos}} \times \underbrace{O(n)}_{\text{CCW Pontos}} + \underbrace{O(nlog(n))}_{\text{Ordenar pontos}} = O(n^3)
      \end{equation}
    }
  \end{overprint}
\end{frame}

\begin{frame}{Algoritmo Jarvis}
  % Add centralized figure
  \begin{center}
    \begin{figure}
      \begin{overprint}
        \onslide<1>\include{figures/algs/jarvis_0}
        \onslide<2>\include{figures/algs/jarvis_1}
        \onslide<3>\include{figures/algs/jarvis_2}
        \onslide<4>\include{figures/algs/jarvis_3}
        \onslide<5>\include{figures/algs/jarvis_4}
        \onslide<6>\include{figures/algs/jarvis_5}
        \onslide<7>\include{figures/algs/jarvis_6}
        \onslide<8->\include{figures/algs/jarvis_9}
      \end{overprint}
    \end{figure}
  \end{center}

  \begin{overprint}
    \onslide<9>{
      \begin{equation}
        \text{Custo} = \underbrace{h}_{\text{Região convexa}} \times (\underbrace{O(n)}_{\text{Próximo Ponto}} + \underbrace{O(n)}_{\text{Pontos dentro}}) = O(hn)
      \end{equation}
    }
  \end{overprint}
\end{frame}

\begin{frame}{Exemplos}

  \begin{overprint}
    \begin{columns}
      \begin{column}{0.5\textwidth}
        \begin{figure}
          \includegraphics<1-2>[width=\textwidth]{./figures/introbs_pointsonly.png}
          \includegraphics<3-4>[width=\textwidth]{./figures/fishdp_pointsonly.png}
          \includegraphics<5-6>[width=\textwidth]{./figures/dog_pointsonly.png}
          \includegraphics<7-8>[width=\textwidth]{./figures/canada_pointsonly.png}
        \end{figure}
      \end{column}
      \begin{column}{0.5\textwidth}
        \begin{figure}
          \includegraphics<2>[width=\textwidth]{./figures/introbs.png}
          \includegraphics<4>[width=\textwidth]{./figures/fishdp.png}
          \includegraphics<6>[width=\textwidth]{./figures/dog.png}
          \includegraphics<8>[width=\textwidth]{./figures/canada.png}
        \end{figure}
      \end{column}
    \end{columns}
  \end{overprint}

\end{frame}

\begin{frame}
\frametitle{Tempo de execução}
% Two columns
\begin{columns}
  \begin{column}{0.4\textwidth}
  \begin{itemize}
    \item Pontos gerados aleatoriamente dentro de um círculo de raio 1
    \item $h \sim \sqrt{n}$
  \end{itemize}
  \end{column}
  \begin{column}{0.6\textwidth}
    \begin{figure}
      \includegraphics[width=0.9\textwidth]{./figures/random_circlw.png}
    \end{figure}
  \end{column}
\end{columns}
  
\end{frame}

\begin{frame}
\frametitle{Tempo de execução}
    \begin{figure}
      \includegraphics[width=0.7\textwidth]{./figures/circles_time.png}
    \end{figure}
\end{frame}

\end{document}
