\documentclass[aspectratio=169,usenames,dvipsnames]{beamer}


\usetheme{default}  % You can choose any other theme you prefer

\title{05 - Algoritmos}
\subtitle{Algoritmos de Triangulação}
\author{Mateus Oliveira de Figueiredo}
\date{03/10/2023}

\usepackage{tikz}
\usetikzlibrary{matrix}
\usepackage{multicol}
\usepackage{algorithm}
\usepackage{algpseudocode}
\usepackage{xcolor}
\usepackage[utf8]{inputenc}
\usepackage[portuguese]{babel}
\usepackage{amsmath} % for "pmatrix" environment  

\usepackage{pgfplots}
\DeclareUnicodeCharacter{2212}{−}
\usepgfplotslibrary{groupplots,dateplot}
\usetikzlibrary{patterns,shapes.arrows, positioning}
\pgfplotsset{compat=newest}

\begin{document}

\begin{frame}
\titlepage
\end{frame}

\foreach \n in {0,...,10} {
\begin{frame}
  % Create a for to generate each slide with event_0, event_1, ..., event_5
\frametitle{Algoritmos de Interseção de Segmentos}
      \begin{figure}
        \includegraphics[width=0.5\textwidth]{figs/event_\n.pdf}
      \end{figure}
\end{frame}

}

\end{document}

  % \begin{overprint}
  % \onslide<1> {Slide 01}
  % \onslide<2> {Slide 02}
  % \onslide<3> {Slide 03}
  % \onslide<4> {Slide 04}
  % \onslide<5> {Slide 05}
  % \onslide<6> {Slide 06}
  % \onslide<7> {Slide 07}
  % \onslide<8> {Slide 08}
  % \onslide<9> {Slide 09}
  % \onslide<10> {Slide 10}
  % \onslide<11> {Slide 11}
  % \onslide<12> {Slide 12}
  % \onslide<13> {Slide 13}
  % \onslide<14> {Slide 14}
  % \onslide<15> {Slide 15}
  % \onslide<16> {Slide 16}
  % \onslide<17> {Slide 17}
  % \onslide<18> {Slide 18}
  % \onslide<19> {Slide 19}
  % \end{overprint}