\documentclass[aspectratio=169,usenames,dvipsnames]{beamer}

\usetheme{default}  % You can choose any other theme you prefer

\title{04 - Algoritmos}
\author{Mateus Oliveira de Figueiredo}
\date{26/09/2023}

\usepackage{tikz}
\usepackage{multicol}
\usepackage{algorithm}
\usepackage{algpseudocode}
\usepackage{xcolor}
\usepackage[utf8]{inputenc}

\usepackage{pgfplots}
\DeclareUnicodeCharacter{2212}{−}
\usepgfplotslibrary{groupplots,dateplot}
\usetikzlibrary{patterns,shapes.arrows, positioning}
\pgfplotsset{compat=newest}

\begin{document}

\begin{frame}
\titlepage
\end{frame}

\begin{frame}
  \frametitle{Problema}

  \onslide<1->{
      Dado um conjunto de pontos no  $\mathbb{R}^2$, encontrar o menor conjunto convexo que contém todos os pontos (fecho convexo).
  }
  \begin{figure}
    \begin{overprint}
      \onslide<1> \include{./figures/points}
      \onslide<2> \include{./figures/points_hull}
    \end{overprint}
  \end{figure}
\end{frame}

\begin{frame}{Algoritmo de Graham}
  % Make 2 columns
  \begin{columns}
    \begin{column}{0.5\textwidth}
      \begin{itemize}
        \onslide<1->{\item Encontrar o ponto mais baixo e a esquerda (pivot)}
        \onslide<3->{\item Ordena os pontos em ordem crescente de ângulo em relação ao pivot}
        \onslide<6->{\item Adiciona os pontos ao fecho convexo, caso não vire para a direita. Caso contrário, remove o ponto anterior.}
      \end{itemize}
    \end{column}
    \begin{column}{0.5\textwidth}
      \include{graham/fig}
    \end{column}
  \end{columns}
  
\end{frame}

\begin{frame}{Exemplos}
  \begin{overprint}
    \begin{columns}
      \begin{column}{0.5\textwidth}
        \begin{figure}
          \includegraphics<1-2>[width=\textwidth]{./figures/cartbs_pointsonly.png}
          \includegraphics<3-4>[width=\textwidth]{./figures/catholebs_pointsonly.png}
          \includegraphics<5-6>[width=\textwidth]{./figures/dog_pointsonly.png}
        \end{figure}
      \end{column}
      \begin{column}{0.5\textwidth}
        \begin{figure}
          \includegraphics<2>[width=\textwidth]{./figures/cartbs.png}
          \includegraphics<4>[width=\textwidth]{./figures/catholebs.png}
          \includegraphics<6>[width=\textwidth]{./figures/dog.png}
        \end{figure}
      \end{column}
    \end{columns}
  \end{overprint}
\end{frame}

\begin{frame}{Algoritmos de Ordenação}

  \vfill

  Algoritmos $O(n^2)$ (pior caso):
  \begin{itemize}
    \item Selection Sort
    \item Insertion Sort
    \item Quick Sort - {\color{red} $O(n log(n))$ (caso médio)}
  \end{itemize}

  \vfill

  Algoritmos $O(n log(n))$ (pior caso):
  \begin{itemize}
    \item Merge Sort
    \item Heap Sort
  \end{itemize}

  \vfill
  
\end{frame}

\begin{frame}
\frametitle{Tempo de execução}
% Two columns
\begin{columns}
  \begin{column}{0.4\textwidth}
  \begin{itemize}
    \item Pontos gerados aleatoriamente dentro de um círculo de raio 1
    \item $h \sim \sqrt{n}$, onde h é o número de vértices do fecho convexo.
  \end{itemize}
  \end{column}
  \begin{column}{0.6\textwidth}
    \begin{figure}
      \includegraphics[width=0.9\textwidth]{./figures/random_merge_1000_1.png}
    \end{figure}
  \end{column}
\end{columns}
  
\end{frame}

\begin{frame}
\frametitle{Tempo de execução}
    \begin{figure}
        \includegraphics[width=0.7\textwidth]{./figures/circle_times.png}
    \end{figure}
\end{frame}

\begin{frame}
\frametitle{Tempo de execução}
    \begin{figure}
        \includegraphics[width=0.7\textwidth]{./figures/circle_times_nlogn.png}
    \end{figure}
\end{frame}

\begin{frame}
\frametitle{Tempo de execução - Graham x Jarvis}
    \begin{figure}
      % include pgf circle_times.pgf here
      \includegraphics[width=0.7\textwidth]{./figures/circle_times_2.png}
    \end{figure}
\end{frame}


\end{document}
