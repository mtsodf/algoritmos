\documentclass[aspectratio=169,usenames,dvipsnames]{beamer}


\usetheme{default}  % You can choose any other theme you prefer

\title{10 - Algoritmos}
\subtitle{Caminho Mínimo}
\author{Mateus Oliveira de Figueiredo}
\date{27/11/2023}

\usepackage{tikz}
\usetikzlibrary{matrix}
\usepackage{multicol}
\usepackage{algorithm}
\usepackage{algpseudocode}
\usepackage{xcolor}
\usepackage[utf8]{inputenc}
\usepackage[portuguese]{babel}
\usepackage{amsmath} % for "pmatrix" environment  
\usepackage{pgffor} 
\usepackage{listings}

\usepackage{pgfplots}
\DeclareUnicodeCharacter{2212}{−}
\usepgfplotslibrary{groupplots,dateplot}
\usetikzlibrary{patterns,shapes.arrows, positioning, arrows}
\usetikzlibrary{graphs, graphs.standard}
\pgfplotsset{compat=newest}

\lstset{
  language=Python,
  basicstyle=\ttfamily\tiny,
  keywordstyle=\color{blue},
  commentstyle=\color{green},
  stringstyle=\color{red},
  stepnumber=1,
  numbersep=10pt,
  showspaces=false,
  showstringspaces=false,
  tabsize=2,
  breaklines=true,
  breakatwhitespace=true,
}

\begin{document}

\begin{frame}
\titlepage
\end{frame}

\begin{frame}
\frametitle{Caminho Mínimo}

Dado um nó fonte encontrar o menor caminho para todos os outros nós.

\vfill
Algoritmos:
\begin{itemize}
  \item Algoritmo de Dijkstra ($O(mlog(m))$)
  \item Algoritmo de Bellman-Ford ($O(mn)$)
  \item Algoritmo de Bellman-Ford com filas ($O(mn)$)
  \item Algoritmo DAGS ($O(m+n)$)
\end{itemize}
\vfill
\end{frame}


\begin{frame}
\frametitle{Proposição Relaxamento}

\vfill
\begin{block}{Proposição}
Seja G um dígrafo com fonte $s \in V$. Seja dist\_para um vetor de distâncias de s para cada vértice de G. 
Este vetor contém as distâncias mínimas se e somente se:
\begin{itemize}
  \item Para cada aresta $u \rightarrow v$ de G, temos que dist\_para[v] $\leq$ dist\_para[u] + peso(u,v).
  \item dist\_para[s] = 0
\end{itemize}
\end{block}
\vfill

\end{frame}

\begin{frame}{Bellamn Ford}
    Relaxar todas as arestas $|V|-1$ vezes.

\end{frame}

\end{document}
